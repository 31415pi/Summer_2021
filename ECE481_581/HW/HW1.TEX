\noindent\textbf{\section{Homework 1}}\\\\
\noindent\rule{\textwidth}{1pt}\\\\
\textbf{\subsection{Problem 1: Determine if the following cyclic circuit is combinational or sequential.}}\\
Give your justification and prove your assertion.\\
\begin{figure}[h]
\includegraphics[scale=0.65]{q1}\
\centering
\end{figure}\\
\tabto{1cm}1) (Traditional Simulation) Can you use the traditional simulation to prove your assertion? How?\\
\tabto{2cm}
\tabto{1cm}2) (Symbolic Simulation) Use symbolic variables to prove your assertion in Boolean logic\\
\tabto{2cm}
\tabto{1cm}3) Design a functional equivalent acyclic circuit with the minimum number of gates. How many gates are used in your design?\\
\\
\tabto{2cm}
\textbf{\subsection{Problem 2: Consider the following cyclic circuit.}}\\
\begin{figure}[h]
\includegraphics[trim={0 0 0 8mm },scale=0.88]{q2}\
\centering
\end{figure}\\
\\
\tabto{1cm}1) Give a detailed discussion on this circuit.\\
\tabto{2cm}
\tabto{1cm}2) What SR inputs cannot be used? Why? Give a detailed reasoning.\\
\tabto{2cm}
\textbf{\subsection{Problem 3: Shannon Decomposition Proof}}\\
\begin{figure}[h]
\includegraphics[scale=1]{q3}\
\centering
\end{figure}\\
\tabto{1cm}1) Use the following Boolean function F = f (A, B, C) to prove the Shannon decomposition in terms of variable C.\\
\tabto{2cm}\\
\\
\tabto{1cm}2) Find one application of the Shannon decomposition. Explain the details.\\
\tabto{2cm}
\textbf{\subsection{Problem 4: Study the following paper and summarize what you learned.}}\\
An agile approach to building RISC-V microprocessors. By Y. Lee, et al. IEEE Micro,\\
Volume: 36, Issue: 2, 2016.\\
