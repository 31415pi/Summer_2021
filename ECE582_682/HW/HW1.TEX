\noindent\textbf{\section{Homework 1}}\\
\noindent\rule{\textwidth}{1pt}\\
\textbf{\subsection{Problem 1. Use the Boolean algebra to prove or answer the questions:}}\\
\tabto{1cm}1) $p \leftrightarrow q = (p \to q)\cap(q\to p)$ \\
\tabto{2cm}\\
\tabto{1cm} 2)$ p \cup q = (p \to q)  \to  q$\\
\tabto{2cm}\\
\tabto{1cm} 3) $(p \leftrightarrow q)’ = p’ \leftrightarrow q$\\
\tabto{2cm}\\
\tabto{1cm} 4) Is $((p  \to  q) \oplus r)  \to  ((p \oplus r)  \to  (q \oplus r))$ a tautology, why? 
\tabto{1cm}Provide the proof or disproof.\\
\tabto{2cm}\\
\tabto{1cm} 5) Is $(p \oplus q) \oplus (p \leftrightarrow q)$ a tautology, why? 
\tabto{1cm} Provide the proof or disproof.\\
\tabto{2cm}\\
\textbf{\subsection{Problem 2. Prove the following assertions:}}\\
\tabto{1cm} 1) If $A \subseteq B and C \subseteq D, then (A \times C) \subseteq (B \times D)$.\\
\tabto{2cm}\\
\tabto{1cm} 2) If A=\{1,2,3,4\}, find $2^{A}=?$\\
\tabto{2cm}\\
\tabto{1cm} 3) $2^{\varnothing}$ = ?\\
\tabto{2cm}\\
\textbf{\subsection{Problem 3. Give an example of Recursion with a detailed execution:}}\\
\\
\textbf{\subsection{Problem 4. Consider the following description:}\\}\\
There are only two formats for photos: round and square. Photos are either color or black and\\
white. Let me tell you about the photo I found yesterday. If the photo is square, then it is a black\\
and white picture. If it is round, it is a digital color picture. If the photo is digital or in black and\\
white, then it is a portrait. If it is a portrait, then it is a picture of my friend.\\
Use variables A, B, C, D, E, F, G to represent the following facts, respectively:\\
\tabto{1cm}A: The photo is in color\\
\tabto{1cm}B: The photo is in black and white\\
\tabto{1cm}C: The photo is square\\
\tabto{1cm}D: The photo is round\\
\tabto{1cm}E: The photo is digital\\
\tabto{1cm}F: The photo is a portrait\\
\tabto{1cm}G: The photo is a picture of my friend\\
\subsubsection{Question 4.1:}\\
Use the logic expressions to represent the following sentences:\\
\tabto{1cm} 1. There are only two formats for Photos: round and square\\
\tabto{2cm}\\
\tabto{1cm} 2. Photos are either color or black and white\\
\tabto{2cm}\\
\tabto{1cm} 3. If the photo is square, then it is a black and white picture.\\
\tabto{2cm}\\
\tabto{1cm} 4. If it is round, it is a digital color picture\\
\tabto{2cm}\\
\tabto{1cm} 5. If the photo is digital or in black and white, then it is a portrait\\
\tabto{2cm}\\
\tabto{1cm} 6. If it is a portrait, then it is a picture of my friend.\\
\tabto{2cm}\\
\subsubsection{{Question 4.2:}}\\
Using logic inference, can you answer the author’s question: Is the photo I found yesterday the\\
picture of my friend?\\
\textbf{\subsection{Problem 5. Consider the following story:}}\\
A says B lies. B says C lies. C says that both A and B are lying.\\
Please use Boolean logic to derive the answer to this question: Who is telling the truth?\\
\textbf{\subsection{Problem 6. Study the following paper and summarize what you learned for one page.}}\\
Aarti Gupta, "Formal Hardware Verification Methods: A Survey", Formal Methods in System\\
Design, Vol. 1, pp. 151-238, 1992.\\

\vspace{5cm}
\noindent\rule{\textwidth}{2pt}\\
\textsc{LaTeX transcription by Mawlee}%Please retain this line if you chose to use this 
